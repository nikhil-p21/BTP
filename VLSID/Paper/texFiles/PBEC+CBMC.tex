The PBECs are sound but not complete. Therefore, all the PBECs 
report that the behaviors ``May Not be equivalent'' once
they fail to prove the equivalence of source and transformed behaviors. Using
the output of CBMC, we can actually make the PBEC more
powerful. In some scenarios, the PBEC can report that the
behaviors are ``Not equivalent'' (instead of ``May Not") along with a
counter-example. Also, in some scenarios, the non-equivalence result reported by
the PBEC can be proved to be a \textit{false negative} and equivalence
checking will proceed further. In the following, we discuss how we can
incorporate the CMBC result to improve the equivalence checking framework.
 \begin{itemize}
  \item {\textbf{Case~1:} \it One of the conditions mentioned in 
  \texttt{\_\_CPROVER\_assume} statement is not 
  satisfiable}: In this case, we report to PBEC that all
  the possible computations represented by \textit{cTrace} are false 
  computations.  Consequently, we need to proceed further in the 
  equivalence checking process.
 \item {\textbf{Case~2:} \it The unwinding assertions are valid and CBMC 
 does not find any counter-example}: This means the values of all the live 
 variables and output variables are the same for both \textit{cTraces}. So the non-equivalence 
 reported by the PBEC may be a false negative. In this case, we need to proceed 
 further in equivalence checking by declaring the corresponding path pair 
 ($\alpha,\beta$) as an equivalent path. This actually helps the PBEC to avoid 
 false negative results during the course of equivalence checking.
 \item {\textbf{Case~3:} \it CBMC reports counter-example for some variables:} 
 This means the data transformation of some variables is not equivalent 
 in the \textit{cTraces}. 
 
 {\textbf{Case~3.1:} \it A mismatch is found for an output variable}:   
 This is surely a non-equivalence case. So the equivalence checker correctly 
 found the non-equivalence of the behaviors. In this case, the PBEC 
 reports that the behaviors are ``Not equivalent'' along with the 
 counter-example. 

 If a mismatch is found only for live variables (which are 
 not output variables), then we cannot conclude definitely that the final 
 outputs of both the behaviors will not be the same. There may be some other 
 operations in the subsequent execution of the FSMDs which will make the 
 behaviors equivalent. Therefore, we need to execute the two programs with the 
 counter-example produced by CBMC and check if the outputs of the two 
 programs are the same or not. 
 
{\textbf{Case~3.2:} \it The outputs are the same}: This is not a 
 non-equivalent case. Consequently, we need to proceed further in the 
 equivalence checking process. 

{\textbf{Case~3.3:} \it The outputs of the two 
 programs are not the same}: This is surely a non-equivalence scenario; 
 in this case, the equivalence checker will report the behaviors are ``Not 
 equivalent'' along with the counter-example.
 \item {\textbf{Case~4:} \it CBMC hits the time limit:} In this case, CBMC  
 has failed to generate a counter-example because of time out. 
 So no counter-example will be provided to the user. The PBEC reports the behaviors ``May Not be equivalent''. 
 \end{itemize}
