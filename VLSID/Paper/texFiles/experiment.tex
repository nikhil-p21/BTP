We have taken the source code of the EVP method and have implemented our counter-example
generation procedure on top of it. Once the EVP method fails to prove the
equivalence, a \textit{cTrace} is automatically generated
using $\mathtt{EQ\_LIST}$ and $\mathtt{C\_LIST}$ of the EVP method as discussed
in Sec.~\ref{Sec:CTrace}.  We have then translated the two corresponding
\textit{cTraces} as an input to CBMC. For our experiment, we used CBMC version
5.8 \cite{Clarke04CBMC}. The benchmarks are taken from \cite{Banerjee14}.  The benchmarks are
run on a 1.8 GHz Intel i5 processor with 8 GB of RAM with a timeout limit of 60
seconds. The results of our experimentation are tabulated in Table~\ref{Table:res}. 
We have manually introduced few changes like addition, multiplication
or subtraction of a constant to some of the variables in the benchmarks 
tabulated in rows 3–6 of Table~\ref{Table:res} so that
source and transformed behaviors become non-equivalent.  For each benchmark, we
have reported the number of paths in the path cover, the number of states in the
source and transformed behaviors, the equivalence decision taken by the EVP method and our method,
the run time in milliseconds (ms) of the EVP method and our method, and the number of lines in the C program given as
an input to CBMC.

For the benchmarks DIFFEQ and LRU, both the EVP method and our method report
equivalence which is denoted as `E' in Table~\ref{Table:res}. The objective is to make sure
that our implementation does not have any side effect on the existing method.
In the benchmarks reported in rows 3--6, the EVP method fails to prove the equivalence of
source and transformed behaviors. It reports that the behaviors ``May Not be
equivalent". This is reported as `MNE' in Table~\ref{Table:res}. In these
cases, CBMC finds the mismatch in the values of output variable and generates a suitable
counter-example with $k=2$ loop unwindings.  Hence, our method concludes that the behaviors are ``Not
equivalent" which is denoted as `NE' in Table~\ref{Table:res}. The time required by our
method is a little high compared to the EVP method in the case of non-equivalence as
we need to run CBMC on the \textit{cTrace}. This experiment shows
that with the help of our counter-example generation scheme a PBEC can take
strong decisions about the non-equivalence of behaviors. Moreover, the counter-example
provided by the PBEC will help the user to debug the root cause of the non-equivalence.

In our second experiment, we try to explore the false negative scenario of the EVP
method.  For this purpose, we have taken the example given in
\cite{Banerjee18} and the result is tabulated in row~7 of 
Table~\ref{Table:res}. This test case involves the inverse operation \cite{Banerjee18}.
 For this test case, the EVP method reports that the behaviors ``May
Not be equivalent". However CBMC  does not generate any counter-example and
case~2 as discussed in Sec~\ref{Sec:EVP+CE} arises here.  CBMC reports that
\textit{cTrace} corresponding to these behaviors are equivalent. Our method
still reports ``May Not be equivalent" since we have not implemented proceed further scenarios
in the EVP. This experiment exposes a false
negative case of the EVP method. It would be an interesting future work to
enhance the EVP to handle the test cases which involves inverse operations.
\begin{table}[!t]
\caption{Experimental Results}
\label{Table:res}
\setlength\tabcolsep{3pt} % default value: 6pt
  \centering
  \begin{tabular}{|l|c|c|c|c|c|c|c|c|}
    \hline
    \multirow{2}{*}{Benchmarks} &
    \multirow{2}{*}{\#Path} &
    \multicolumn{2}{c|}{\#State} &
    \multicolumn{2}{c|}{Decision} &
    \multicolumn{2}{c|}{Time (ms)} &
    \multirow{2}{*}{Lines}\\\cline{3-8}
    &&$M_0$&$M_1$&EVP&Our&EVP&Our&\\\hline
    DIFFEQ&3&15&9&E&E&25&25&-\\  
    LRU&39&33&32&E&E&1038&1038&-\\  
    DCT&1&8&16&MNE&NE&85&766&185\\ %output change
    PERFECT&7&6&4&MNE&NE&56&227&74\\  %live Variable change
    MODN&9&8&9&MNE&NE&66&890&137\\ %live Variable Change
    GCD&11&8&4&MNE&NE&31&100&97\\\hline 
    Test Case\cite{Banerjee18}&6&5&5&MNE&MNE&20&26&32\\\hline 
  \end{tabular}
  \end{table} 





