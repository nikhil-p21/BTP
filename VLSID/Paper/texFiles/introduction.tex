High-level synthesis (HLS) is the process of translating a behavioral
description into a Register Transfer Level (RTL) description \cite{Gajski92}.
HLS tools are large and complex software systems and are very often written
without formally proving their correctness. Many path-based
approaches
\cite{KimKM04,KimM08,HuLL16,Banerjee14,Karfa12,Karfa08,Chouksey18} have
been proposed for verification of the scheduling phase of HLS
where each behavior is represented by a finite state machine with datapaths (FSMD) \cite{Gajski92}. In
general, path-based approaches decompose each FSMD into a 
finite set of finite paths and the equivalence of FSMDs
is established by showing path level equivalence between two  FSMDs.
In the case of non-equivalence, these approaches do not provide information 
sufficient for debugging the issue. 
A counter-example which will demonstrate the non-equivalence
between the input behavior to HLS (i.e., source behavior) and 
the scheduled behavior generated by HLS (i.e., transformed behavior) will add significant value to
the adoption of such PBECs. In this case, PBEC can report ``Not
equivalent" instead of ``May Not be equivalent".  Equivalence checking of programs is
an undecidable  problem in general. Therefore it is possible that a PBEC may produce a
\textit{false negative} result, i.e., a PBEC may report that 
two behaviors ``May Not be equivalent'' but these two behaviors are actually equivalent.
The process of generating a counter-example helps to identify some
false negative cases of a PBEC.  
Thus, a counter-example generation procedure helps to improve the performance of a PBEC. 

Specifically, the contributions of the paper are as follows: 
\begin{enumerate} 
\item We show how the equivalence information of the enhanced value propagation
(EVP) based PBEC \cite{Chouksey18} can be used to find a $cTrace$ in the case
of non-equivalence reported by the PBEC.
\item We show how the CBMC \cite{Clarke04CBMC} tool can be used to find a suitable
counter-example for a given $cTrace$. 
\item We show how to improve the performance of the PBEC using this
counter-example generation framework.
\item An enhanced version of PBEC \cite{Chouksey18} after incorporating our
counter-example generation scheme is also presented.  
\end{enumerate} 
To the best of our knowledge, this is the first work which
reports a $cTrace$ in the case of non-equivalence and uses it to produce a
counter-example and improve the performance of PBECs during verification of 
the scheduling phase of HLS.

The rest of this paper is organized as follows. Section~\ref{Sec:prelim} describes
an FSMD model and the EVP method.  Section~\ref{Sec:CTrace} focuses on
\textit{cTrace} generation.  Section~\ref{Sec:CE} presents how that \textit{cTrace} can be used
to produce a counter-examples using CBMC.
Section~\ref{Sec:incorpoResult} and \ref{Sec:EVP+CE} finally delve into how
current PBECs can be enhanced by incorporating our counter-example generation
technique. Experimental results are given in Sec.~\ref{Sec:Exp}.
Section~\ref{Sec:Rework} contains a summary of the related work.
Section~\ref{Sec:conclusion} concludes the paper.
