Suppose the source behavior and the transformed behavior are
represented as FSMDs $M_0$ and $M_1$, respectively. Let us assume that the
PBEC fails to find an equivalent for
the path $\alpha$ of $M_0$. We now discuss how to generate a
unique computation starting from the reset state that leads to the path
$\alpha$. It may be noted that the EVP method maintains two lists:
$\mathtt{EQ\_LIST}$ contains equivalent path pairs explored so far and
$\mathtt{C\_LIST}$ contains candidates for conditionally equivalent path pairs.
In the EVP method $\mathtt{C\_LIST}$ is obtained in a
depth first search (DFS) manner. So, if we traverse backward from the start
state of $\alpha$, we will obtain a sequence of paths from the set
$\mathtt{C\_LIST}$. \textit{This trace would always be a unique trace.} Let the
sequence be $\langle p_{0j},p_{0j+1},\dots,p_{0k},\alpha \rangle$ in FSMD
$M_0$. The segment of the FSMD $M_0$ from the reset state $q_{00}$ to the start
state of $p_{0j}$, (say $p_{0j}^s$,) is already proved to be equivalent to its
corresponding part in FSMD $M_1$. However, there may be many paths from
$q_{00}$ to $p_{0j}^s$. For our purpose, we can choose one of the paths  from
this segment. Let us choose the sequence $\langle
p_{00},p_{01},\dots,p_{0i}\rangle$ where $p_{00}$ starts from the state
$q_{00}$ and the path $p_{0i}$ ends at $p_{0j}^s$. Therefore, the sequence
$\mathit{cTrace}=\langle
p_{00},p_{01},\dots,p_{0i},p_{0j},p_{0j+1},\dots,p_{0k},\alpha\rangle$ is the
\textit{cTrace} in the FSMD $M_0$ that we are interested in. From
$\mathtt{EQ\_LIST}$, we will obtain the paths corresponding to
$p_{00},p_{01},\dots,p_{0i}$ in FSMD $M_1$. Let the corresponding paths be
$p_{10},p_{11},\dots,p_{1i}$, respectively. Similarly, the corresponding paths
of $p_{0j},p_{0j+1},\dots,p_{0k}$ in the FSMD $M_1$ can be found using
$\mathtt{C\_LIST}$. Let the corresponding paths be
$p_{1j},p_{1j+1},\dots,p_{1k}$, respectively. The \textit{potential}
corresponding path of $\alpha$ can also be obtained in the FSMD $M_1$; let it
be $\beta$.  The EVP method identifies the potential candidate for equivalence,
$\beta$, in $M_1$ in most of the cases (see~\cite{Chouksey18} for details). It
fails to find $\beta$ only if there does not exist any path from the
corresponding state in $M_1$ whose condition of execution matches even partially
with that of $\alpha$. In this case, we can take any path from the
corresponding state in $M_1$. Therefore, the corresponding $cTrace$ in FSMD
$M_1$ is $~\langle
p_{10},p_{11},\dots,p_{1i},p_{1j},p_{1j+1},\dots,p_{1k},\beta\rangle$.
\begin{example} 
Consider the input behavior $M_0$ and its transformed behavior
$M_1$ shown in Fig.~\ref{Fig:NonEqui}. The operation $x\Leftarrow 5$, a loop
invariant for input behavior $M_0$, is placed after the loop body in the
transformed behavior $M_1$. Note that the input behavior $M_0$ and the
transformed behavior $M_1$, shown in Fig.~\ref{Fig:NonEqui}, are not equivalent
since there is mismatch in values of the  $\mathit{out}$ variable. The EVP
method reports that behaviors ``May Not be equivalent". The EVP method also
reports that the path pairs $(p_{00},p_{10})$ and $(p_{01},p_{11})$ are
U-equivalent, the path pair $(p_{02},p_{12})$ is a candidate for C-equivalence and the path pair
$(p_{03},p_{13})$ is not equivalent.  During the course of equivalence checking
the EVP method stores these U-equivalent and candidate for C-equivalent path pairs in the
$\mathtt{EQ\_LIST}$ and $\mathtt{C\_LIST}$ list, respectively. 
As explained in Sec.~\ref{Sec:CTrace}, using these lists the
generated \textit{cTrace} of $M_0$ and $M_1$ is shown in Fig.~\subref*{SubFig:CM0}
and Fig.~\subref*{SubFig:CM1}, respectively.
\end{example} 
