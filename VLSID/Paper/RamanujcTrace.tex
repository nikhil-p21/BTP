\documentclass[conference,10pt]{IEEEtran}
\usepackage{cite}
\usepackage{textcomp}
\usepackage[pdftex]{graphicx}
\usepackage{amsmath}
\usepackage{amsthm}
\usepackage{mathtools}
\usepackage{tikz}
\usepackage[boxed,ruled,linesnumbered,longend]{algorithm2e}
\usetikzlibrary{shapes,arrows,chains}
\usepackage{alltt}
\usepackage{multirow}
\usepackage{enumitem}
\usepackage{verbatim}
\usepackage{listings}
\def\BibTeX{{\rm B\kern-.05em{\sc i\kern-.025em b}\kern-.08em
    T\kern-.1667em\lower.7ex\hbox{E}\kern-.125emX}}
\SetCommentSty{mycommfont}
\usetikzlibrary{shapes,arrows,chains,calc,trees,backgrounds,fit,decorations.markings,automata,positioning}
\usepackage[boxed,ruled,linesnumbered,longend]{algorithm2e}
\newcommand\mycommfont[1]{\footnotesize\textcolor{blue}{#1}}
\SetCommentSty{mycommfont}
\ifCLASSOPTIONcompsoc
\usepackage[caption=false,font=normalsize,labelfont=sf,textfont=sf,subrefformat=parens,labelformat=parens]{subfig}
\else
\usepackage[caption=false,font=footnotesize,subrefformat=parens,labelformat=parens]{subfig}
\fi
\newtheorem{lemma}{Lemma}
\newtheorem{theorem}{Theorem}
\newtheorem{definition}{Definition}
\newtheorem{example}{Example}
\newtheorem{proposition}{Proposition}
\def \path {\Longrightarrow}
\begin{document} 
\title{Improving Performance of a Path-Based Equivalence Checker using Counter-Examples}
\author{\IEEEauthorblockN{Ramanuj Chouksey\IEEEauthorrefmark{1},
Chandan Karfa\IEEEauthorrefmark{2} and
Purandar Bhaduri\IEEEauthorrefmark{3}}
\IEEEauthorblockA{Department of Computer Science and Engineering,\\
Indian Institute of Technology Guwahati, 781039, India\\
Email: \IEEEauthorrefmark{1}r.chouksey@iitg.ac.in,
\IEEEauthorrefmark{2}ckarfa@iitg.ac.in,
\IEEEauthorrefmark{3}pbhaduri@iitg.ac.in}}
\maketitle 
\begin{abstract}  
Path-based equivalence checkers (PBECs) have been successfully applied for verification of 
programs from diverse domains and at various stages of high-level synthesis. 
These verifiers can be sound but not complete. Therefore, non-equivalence cases 
require further investigation of the two programs being compared by some human expert.  
In this work, we show how a counter-trace (\textit{cTrace}) can be generated in the case of 
non-equivalence reported by the PBEC. 
We show how a Bounded Model Checker (CBMC) can be
used to find suitable initialization values for input variables (i.e., a
counter-example) for a given \textit{cTrace}.
With our counter-example generation framework, we show how a strong non-equivalence decision 
can be taken in a PBEC. We also show that some false negative cases of the PBEC can also be 
revealed using this framework. Experimental results demonstrate the usefulness
of our method.

\end{abstract}
\begin{IEEEkeywords} Equivalence Checking, Finite State Machine with Datapath (FSMD), CBMC, Counter-example Generation.  
\end{IEEEkeywords}
\IEEEpeerreviewmaketitle
\section{Introduction}\label{Sec:intro}
High-level synthesis (HLS) is the process of translating a behavioral
description into a Register Transfer Level (RTL) description \cite{Gajski92}.
HLS tools are large and complex software systems and are very often written
without formally proving their correctness. Many path-based
approaches
\cite{KimKM04,KimM08,HuLL16,Banerjee14,Karfa12,Karfa08,Chouksey18} have
been proposed for verification of the scheduling phase of HLS
where each behavior is represented by a finite state machine with datapaths (FSMD) \cite{Gajski92}. In
general, path-based approaches decompose each FSMD into a 
finite set of finite paths and the equivalence of FSMDs
is established by showing path level equivalence between two  FSMDs.
In the case of non-equivalence, these approaches do not provide information 
sufficient for debugging the issue. 
A counter-example which will demonstrate the non-equivalence
between the input behavior to HLS (i.e., source behavior) and 
the scheduled behavior generated by HLS (i.e., transformed behavior) will add significant value to
the adoption of such PBECs. In this case, PBEC can report ``Not
equivalent" instead of ``May Not be equivalent".  Equivalence checking of programs is
an undecidable  problem in general. Therefore it is possible that a PBEC may produce a
\textit{false negative} result, i.e., a PBEC may report that 
two behaviors ``May Not be equivalent'' but these two behaviors are actually equivalent.
The process of generating a counter-example helps to identify some
false negative cases of a PBEC.  
Thus, a counter-example generation procedure helps to improve the performance of a PBEC. 

Specifically, the contributions of the paper are as follows: 
\begin{enumerate} 
\item We show how the equivalence information of the enhanced value propagation
(EVP) based PBEC \cite{Chouksey18} can be used to find a $cTrace$ in the case
of non-equivalence reported by the PBEC.
\item We show how the CBMC \cite{Clarke04CBMC} tool can be used to find a suitable
counter-example for a given $cTrace$. 
\item We show how to improve the performance of the PBEC using this
counter-example generation framework.
\item An enhanced version of PBEC \cite{Chouksey18} after incorporating our
counter-example generation scheme is also presented.  
\end{enumerate} 
To the best of our knowledge, this is the first work which
reports a $cTrace$ in the case of non-equivalence and uses it to produce a
counter-example and improve the performance of PBECs during verification of 
the scheduling phase of HLS.

The rest of this paper is organized as follows. Section~\ref{Sec:prelim} describes
an FSMD model and the EVP method.  Section~\ref{Sec:CTrace} focuses on
\textit{cTrace} generation.  Section~\ref{Sec:CE} presents how that \textit{cTrace} can be used
to produce a counter-examples using CBMC.
Section~\ref{Sec:incorpoResult} and \ref{Sec:EVP+CE} finally delve into how
current PBECs can be enhanced by incorporating our counter-example generation
technique. Experimental results are given in Sec.~\ref{Sec:Exp}.
Section~\ref{Sec:Rework} contains a summary of the related work.
Section~\ref{Sec:conclusion} concludes the paper.

\section{The FSMD model and path-based equivalence checking}\label{Sec:prelim}
This section briefly explains the FSMD model and the EVP method\cite{Chouksey18}.
The details can be found in \cite{Chouksey18,Banerjee14}.

An FSMD is defined as a 7-tuple $\langle Q, q_0, I, V, O, f:Q\times 2^S \rightarrow Q,
h:Q\times 2^S \rightarrow U \rangle$, where $Q$ is the finite set of
states, $q_0$ is the reset state, $I$ is the set of input variables, $V$ is the
set of storage variables, $O$ is the set of output variables, $f$ is the state
transition function, $h$ is the update function of the output and the storage
variables. Here $U$ represents a set of storage and output assignments and $S$
represents a set of relations over arithmetic expressions and Boolean literals.

A computation of an FSMD is a finite walk from the reset state $q_0$ to itself,
and $q_0$ should not occur in between.  The papers \cite{Chouksey18,Banerjee14}
breaks down an FSMD into smaller segments by introducing cutpoints so that each
loop in an FSMD is cut at at least one cutpoint.  A {\em path} $\alpha $ is a
finite sequence of states from a cutpoint to another cutpoint without an
intermediate occurrence of a cutpoint.  The {\em condition of execution
$R_{\alpha}$} of a path $\alpha$ is a logical expression over $I \bigcup V$
such that $R_\alpha$ is satisfied by the (initial) data state of the path iff
the path $\alpha$ is traversed.  The {\em data transformation $r_{\alpha}$} of
a path $\alpha$ over $V$ is the tuple $\langle s_\alpha, \theta_\alpha
\rangle$; the first member $s_\alpha$ represents the value of the variables
$v_i$ after the execution of the path in terms of the initial data state of the
path; the second member $\theta_\alpha$ represents the output list along the
path $\alpha$. Two paths $\alpha$ and $\beta$ are equivalent denoted by
$\alpha\simeq\beta$ if $R_{\alpha}\equiv R_{\beta}$ and $r_{\alpha}=r_{\beta}$.
A finite set of paths $P=\{ \alpha_1, \alpha_2, \ldots, \alpha_k \}$ is said to
be a path cover of an FSMD $M$ if any computation $\mu$ of $M$ can be looked
upon as a concatenation of paths from $P$ \cite{Floyd67}. 

An FSMD $M_0$ is contained in another FSMD $M_1$, symbolically $M_0 \sqsubseteq
M_1$, if there exists a finite path cover $P_0=\{ \alpha_1, \alpha_2, \ldots,
\alpha_l \}$ of $M_0$ for which there exists a set $P_1 = \{ \beta_1, \beta_2,
\ldots,$ $\beta_l \}$ of paths of $M_1$ such that $\alpha_i \simeq \beta_i$,
$1\leq i \leq l$.  Two FSMDs $M_0$ and $M_1$ are said to be
computationally equivalent ($M_0 \equiv M_1$), if $M_0 \sqsubseteq M_1$ and
$M_1 \sqsubseteq M_0$.

The EVP method \cite{Chouksey18} is based on propagating the mismatched
variable values (as \textit{propagated vectors}) over a path to the subsequent
paths until the values match or the final path segments end in the reset
state without a match.  During the course of equivalence checking of
two behaviors, two paths, $\alpha$ and $\beta$ say, (one from each behavior)
are compared with respect to their corresponding propagated vectors for finding
path equivalence. If $R_{\beta}\equiv R_{\alpha}$ and $r_{\beta}=r_{\alpha}$ ,
then these paths are declared as unconditionally equivalent (U-equivalent,
represented as $\alpha \simeq \beta$); if some mismatch is detected in data
transformation, then they are declared to be conditionally equivalent
(C-equivalent, represented as $\alpha \simeq_c \beta$), if their final
state-pair always eventually lead to some U-equivalent paths; otherwise they
are declared to be not equivalent.

\begin{figure*}[!t]
\subfloat[Source behavior $M_0$]
{	
		\label{SubFig:M0}
			\linespread{1.5}
			\scalebox{0.65}{
				\begin{tikzpicture}[line width=1.5pt,place/.style={circle,draw}]
				%%%%%%%%%%%%%%%%%%%%%%%%%%%M0%%%%%%%%%%%%%%%%%%%%%%
					\node[fill=gray!20] (1) at (0,1)   [place] {$q_{00}$};
					\node[fill=gray!20] (2) at (0,-2)   [place] {$q_{01}$};
					\node (3) at (-1.5,-5)   [place] {$q_{02}$};
					\node (5) at (1.5,-5)   [place] {$q_{03}$};
					
					\draw [>=latex,->,draw=green](1) to [align=center] node[pos=0.5,left] {\footnotesize $n\geq 0/$\\[-0.3cm] \footnotesize $i\Leftarrow 0,$\\[-0.3cm]\footnotesize $x\Leftarrow 0,$\\[-0.3cm]\footnotesize $y\Leftarrow 0$} 
					node[pos=0.4,right]{$p_{01}$} (2);
					
					\draw [>=latex,->,draw=yellow](2)  [align=center] to  node[pos=0.6,right] {\footnotesize$i\leq n/$\\[-0.2cm]\footnotesize $\boxed{\mathbf{x\Leftarrow 5}},$\\[-0.3cm]\footnotesize $y\Leftarrow y+5$}(3);
					
					\draw [>=latex,->, draw=yellow](3) [align=center] .. controls (-3,-2) and (-0.8,-2) ..  node[pos=0.3,left] {\rotatebox[origin=c]{90}{\footnotesize -/\footnotesize$i\Leftarrow i+1$}} 	node[pos=0.3,right,xshift=0.3cm]{$p_{02}$}(2);
					
					\draw [>=latex,->,draw=red](2)  [align=center] --node[pos=0.5,right]{\footnotesize$\neg i\leq n/$\\[-0.2cm]\footnotesize{\boxed{\mathbf{out\Leftarrow x+y}}}}
					node[pos=0.8,right]{$p_{03}$}(5);
					
					\draw [>=latex,->,draw=green](1)  [align=center] .. controls (2,1) and (5,-4) ..  
					node[pos=0.4,left] {\footnotesize $\neg n\geq 0$/\\[-0.3cm]\footnotesize$out\Leftarrow -1$} node[pos=0.4,right]{$p_{00}$} (5);
									
				\end{tikzpicture}
				}
	}	\hspace{-2cm}
	\subfloat[Transformed behavior $M_1$]
	{	
			\label{SubFig:M1}
			\linespread{1.5}
			\scalebox{0.65}{
				\begin{tikzpicture}[line width=1.5pt,place/.style={circle,draw}]
				%%%%%%%%%%%%%%%%%%%%%%%%%%%M0%%%%%%%%%%%%%%%%%%%%%%
					\node[fill=gray!20] (1) at (0,1)   [place] {$q_{10}$};
					\node[fill=gray!20] (2) at (0,-2)   [place] {$q_{11}$};
					\node (3) at (-1.5,-5)   [place] {$q_{12}$};
					\node (5) at (1.5,-5)   [place] {$q_{13}$};
					
					\draw [>=latex,->,draw=green](1) to [align=center] node[pos=0.5,left] {\footnotesize $n\geq 0/$\\[-0.3cm] \footnotesize $i\Leftarrow 0,$\\[-0.3cm]\footnotesize $x\Leftarrow 0,$\\[-0.3cm]\footnotesize $y\Leftarrow 0$}
					node[pos=0.4,right]{$p_{11}$}
					(2);
					
					\draw[>=latex,->,draw=yellow](2)  [align=center] to  node[pos=0.6,right] {\footnotesize$i\leq n/$\\[-0.3cm]\footnotesize $y\Leftarrow y+5$}(3);
					
					\draw [>=latex,->,draw=yellow](3) [align=center] .. controls (-3,-2) and (-0.8,-2) ..  
					node[pos=0.3,left] {\rotatebox[origin=c]{90}{\footnotesize -/\footnotesize$i\Leftarrow i+1$}}
					node[pos=0.3,right,xshift=0.3cm]{$p_{12}$}(2);
					
					\draw [>=latex,->,draw=red](2)  [align=center] --node[pos=0.3,right,]{\footnotesize$\neg i\leq n/$\\[-0.2cm]\footnotesize $\boxed{\mathbf{x\Leftarrow 5}},$\\[-0.1cm]\footnotesize
					${\boxed{\mathbf{out\Leftarrow x+y+1}}}$}
					node[pos=0.8,right]{$p_{13}$}
					(5);
					
					\draw [>=latex,->,draw=green](1)  [align=center] .. controls (2,1) and (5,-4) ..  
					node[pos=0.4,left] {\footnotesize $\neg n\geq 0$/\\[-0.3cm]\footnotesize$out\Leftarrow -1$} node[pos=0.4,right]{$p_{10}$} (5);
					
				\end{tikzpicture}
				}
	}	\hspace{-1.5cm}			
	\subfloat[\textit{cTrace} of $M_0$]
	{
			\label{SubFig:CM0}	
			\linespread{1.5}
			\scalebox{0.65}{
				\begin{tikzpicture}[line width=1.5pt,place/.style={circle,draw}]
				%%%%%%%%%%%%%%%%%%%%%%%%%%%M0%%%%%%%%%%%%%%%%%%%%%%
					\node[fill=gray!20] (1) at (0,1)   [place] {$q_{00}$};
					\node[fill=gray!20] (2) at (0,-2)   [place] {$q_{01}$};
					\node (3) at (-1.5,-5)   [place] {$q_{02}$};
					\node (5) at (1.5,-5)   [place] {$q_{03}$};
					
					\draw [>=latex,->,draw=green](1) to [align=center] node[pos=0.5,left] 
					{\footnotesize $n\geq 0/$\\[-0.3cm] \footnotesize $i\Leftarrow 
					0,$\\[-0.3cm]\footnotesize $x\Leftarrow 0,$\\[-0.3cm]\footnotesize 
					$y\Leftarrow 0$} 
					node[pos=0.4,right]{$p_{01}$} (2);
					
					\draw [>=latex,->,draw=yellow](2)  [align=center] to  
					node[pos=0.6,right] {\footnotesize$i\leq n/$\\[-0.2cm]\footnotesize 
					$\boxed{\mathbf{x\Leftarrow 5}},$\\[-0.3cm]\footnotesize $y\Leftarrow 
					y+5$}(3);
					
					\draw [>=latex,->, draw=yellow](3) [align=center] .. controls (-3,-2) 
					and (-0.8,-2) ..  node[pos=0.3,left] 
					{\rotatebox[origin=c]{90}{\footnotesize -/\footnotesize$i\Leftarrow 
					i+1$}} 	node[pos=0.3,right,xshift=0.3cm]{$p_{02}$}(2);
					
					\draw [>=latex,->,draw=red](2)  [align=center] 
					--node[pos=0.5,right]{\footnotesize$\neg i\leq 
					n/$\\[-0.2cm]\footnotesize{\boxed{\mathbf{out\Leftarrow x+y}}}}
					node[pos=0.8,right]{$p_{03}$}(5);				
				\end{tikzpicture}
				}
	}	\hspace{-1.0cm}
	\subfloat[\textit{cTrace} of $M_1$]
	{
			\label{SubFig:CM1}	
			\linespread{1.5}
			\scalebox{0.65}{
				\begin{tikzpicture}[line width=1.5pt,place/.style={circle,draw}]
				%%%%%%%%%%%%%%%%%%%%%%%%%%%M0%%%%%%%%%%%%%%%%%%%%%%
					\node[fill=gray!20] (1) at (0,1)   [place] {$q_{10}$};
					\node[fill=gray!20] (2) at (0,-2)   [place] {$q_{11}$};
					\node (3) at (-1.5,-5)   [place] {$q_{12}$};
					\node (5) at (1.5,-5)   [place] {$q_{13}$};
					
					\draw [>=latex,->,draw=green](1) to [align=center] node[pos=0.5,left] 
					{\footnotesize $n\geq 0/$\\[-0.3cm] \footnotesize $i\Leftarrow 
					0,$\\[-0.3cm]\footnotesize $x\Leftarrow 0,$\\[-0.3cm]\footnotesize 
					$y\Leftarrow 0$}
					node[pos=0.4,right]{$p_{11}$}
					(2);
					
					\draw[>=latex,->,draw=yellow](2)  [align=center] to  
					node[pos=0.6,right] {\footnotesize$i\leq n/$\\[-0.3cm]\footnotesize 
					$y\Leftarrow y+5$}(3);
					
					\draw [>=latex,->,draw=yellow](3) [align=center] .. controls (-3,-2) 
					and (-0.8,-2) ..  
					node[pos=0.3,left] {\rotatebox[origin=c]{90}{\footnotesize 
					-/\footnotesize$i\Leftarrow i+1$}}
					node[pos=0.3,right,xshift=0.3cm]{$p_{12}$}(2);
					
					\draw [>=latex,->,draw=red](2)  [align=center] 
					--node[pos=0.3,right,]{\footnotesize$\neg i\leq 
					n/$\\[-0.2cm]\footnotesize $\boxed{\mathbf{x\Leftarrow 
					5}},$\\[-0.1cm]\footnotesize
					${\boxed{\mathbf{out\Leftarrow x+y+1}}}$}
					node[pos=0.8,right]{$p_{13}$}
					(5);
				\end{tikzpicture}
				}
	}
	\caption[]{Counter-trace Genegartion} 
					\label{Fig:NonEqui}
	\end{figure*}
\begin{comment}
	\begin{figure}[!t]
			\scalebox{0.7}{
		\begin{tikzpicture}[list/.style={rectangle split, rectangle split parts=2,
						    draw, rectangle split horizontal}, >=stealth]
						  
						  \node (A) {EQ\_LIST = };
						 	\node[list,right=of A,xshift=-1cm,] (A2) {$(P_{00},P_{10})$};
						  \node[list,right=of A2] (A3) {$(P_{01},P_{11})$};
							\node[right=of A3,draw,inner sep=6pt] (A31) {};
				
							\node[below=of A,yshift=0.6cm,xshift=0.145cm] (B) {C\_LIST = 
							};			  
							\node[list,right=of B,xshift=-1cm,] (B2) {$(P_{02},P_{12})$};
							\node[right=of B2,draw,inner sep=6pt] (B21) {};
						  				  
						 
						  \draw(A31.north east) -- (A31.south west);
						  \draw(A31.north west) -- (A31.south east);
						
							\draw(B21.north east) -- (B21.south west);
							\draw(B21.north west) -- (B21.south east);
						  			
						  
						  \draw[*->] let \p1 = (A2.two), \p2 = (A2.center) in (\x1,\y2) -- 
						  (A3);
						  \draw[*->] let \p1 = (A3.two), \p2 = (A3.center) in (\x1,\y2) -- 
						  (A31);
						  \draw[*->] let \p1 = (B2.two), \p2 = (B2.center) in (\x1,\y2) -- 
						  (B21);
						\end{tikzpicture}
						}
			\caption[]{List maintained during equivalence checking}
			\label{Fig:CList}
		\end{figure}															
\end{comment}

\section{Counter-trace generation \label{Sec:CTrace}}
Suppose the source behavior and the transformed behavior are
represented as FSMDs $M_0$ and $M_1$, respectively. Let us assume that the
PBEC fails to find an equivalent for
the path $\alpha$ of $M_0$. We now discuss how to generate a
unique computation starting from the reset state that leads to the path
$\alpha$. It may be noted that the EVP method maintains two lists:
$\mathtt{EQ\_LIST}$ contains equivalent path pairs explored so far and
$\mathtt{C\_LIST}$ contains candidates for conditionally equivalent path pairs.
In the EVP method $\mathtt{C\_LIST}$ is obtained in a
depth first search (DFS) manner. So, if we traverse backward from the start
state of $\alpha$, we will obtain a sequence of paths from the set
$\mathtt{C\_LIST}$. \textit{This trace would always be a unique trace.} Let the
sequence be $\langle p_{0j},p_{0j+1},\dots,p_{0k},\alpha \rangle$ in FSMD
$M_0$. The segment of the FSMD $M_0$ from the reset state $q_{00}$ to the start
state of $p_{0j}$, (say $p_{0j}^s$,) is already proved to be equivalent to its
corresponding part in FSMD $M_1$. However, there may be many paths from
$q_{00}$ to $p_{0j}^s$. For our purpose, we can choose one of the paths  from
this segment. Let us choose the sequence $\langle
p_{00},p_{01},\dots,p_{0i}\rangle$ where $p_{00}$ starts from the state
$q_{00}$ and the path $p_{0i}$ ends at $p_{0j}^s$. Therefore, the sequence
$\mathit{cTrace}=\langle
p_{00},p_{01},\dots,p_{0i},p_{0j},p_{0j+1},\dots,p_{0k},\alpha\rangle$ is the
\textit{cTrace} in the FSMD $M_0$ that we are interested in. From
$\mathtt{EQ\_LIST}$, we will obtain the paths corresponding to
$p_{00},p_{01},\dots,p_{0i}$ in FSMD $M_1$. Let the corresponding paths be
$p_{10},p_{11},\dots,p_{1i}$, respectively. Similarly, the corresponding paths
of $p_{0j},p_{0j+1},\dots,p_{0k}$ in the FSMD $M_1$ can be found using
$\mathtt{C\_LIST}$. Let the corresponding paths be
$p_{1j},p_{1j+1},\dots,p_{1k}$, respectively. The \textit{potential}
corresponding path of $\alpha$ can also be obtained in the FSMD $M_1$; let it
be $\beta$.  The EVP method identifies the potential candidate for equivalence,
$\beta$, in $M_1$ in most of the cases (see~\cite{Chouksey18} for details). It
fails to find $\beta$ only if there does not exist any path from the
corresponding state in $M_1$ whose condition of execution matches even partially
with that of $\alpha$. In this case, we can take any path from the
corresponding state in $M_1$. Therefore, the corresponding $cTrace$ in FSMD
$M_1$ is $~\langle
p_{10},p_{11},\dots,p_{1i},p_{1j},p_{1j+1},\dots,p_{1k},\beta\rangle$.
\begin{example} 
Consider the input behavior $M_0$ and its transformed behavior
$M_1$ shown in Fig.~\ref{Fig:NonEqui}. The operation $x\Leftarrow 5$, a loop
invariant for input behavior $M_0$, is placed after the loop body in the
transformed behavior $M_1$. Note that the input behavior $M_0$ and the
transformed behavior $M_1$, shown in Fig.~\ref{Fig:NonEqui}, are not equivalent
since there is mismatch in values of the  $\mathit{out}$ variable. The EVP
method reports that behaviors ``May Not be equivalent". The EVP method also
reports that the path pairs $(p_{00},p_{10})$ and $(p_{01},p_{11})$ are
U-equivalent, the path pair $(p_{02},p_{12})$ is a candidate for C-equivalence and the path pair
$(p_{03},p_{13})$ is not equivalent.  During the course of equivalence checking
the EVP method stores these U-equivalent and candidate for C-equivalent path pairs in the
$\mathtt{EQ\_LIST}$ and $\mathtt{C\_LIST}$ list, respectively. 
As explained in Sec.~\ref{Sec:CTrace}, using these lists the
generated \textit{cTrace} of $M_0$ and $M_1$ is shown in Fig.~\subref*{SubFig:CM0}
and Fig.~\subref*{SubFig:CM1}, respectively.
\end{example} 

\section{Counter example generation using counter-trace\label{Sec:CE}}
\begin{figure}[!h]
\begin{center}
\begin{lstlisting}[language=C,
,escapechar = ~,numbers=left,
tabsize=2,basicstyle = \linespread{1.2}\ttfamily\scriptsize, linewidth = 
.6\linewidth,xleftmargin=.04\textwidth] 
#include<assert.h>
void main()
{
  int i_s,x_s,y_s,n,out_s;
  int i_t,x_t,y_t,out_t;
  __CPROVER_assume(n>=0);
  assert(!(n>=0));
  // cTrace for M0
  if(n>=0)
  {
    i_s=0;x_s=0;y_s=0;
    __CPROVER_assume(i_s<=n);
    assert(!(i_s<=n));
    while(i_s<=n)
    {
      x_s=5;
      y_s=y_s+5;
      i_s=i_s+1;
    }
    out_s=x_s+y_s;
  }
  //cTrace for M1
  if(n>=0)
  {
    i_t=0;x_t=0;y_t=0;
    __CPROVER_assume(i_t<=n);
    assert(!(i_t<=n));
    while(i_t<=n)
    {
      y_t=y_t+5;
      i_t=i_t+1;
    }
    x_t=5;
    out_t=x_t+y_t+1;
  }
  assert(x_s = x_t);// Live Variable
  assert(y_s = y_t);// Live Variable
  assert(out_s = out_t);// Output Variable
}
\end{lstlisting}
\end{center}
%\caption{CBMC Input}
\caption{CBMC input for the \textit{cTraces} shown in Fig.~\protect\subref*{SubFig:CM0} and Fig.~\protect\subref*{SubFig:CM1}}	
\label{Fig:CBMCInput}
\end{figure}


\begin{figure*}[!t]
 \centering
 \scalebox{0.34}{
 \begin{tikzpicture}[%
      >=triangle 60,              % Nice arrows; your taste may be different
      node distance=7mm and 4mm, % Global setup of box spacing
      every join/.style={norm},   % Default linetype for connecting boxes
 	   font=\Large,auto
      ]
  \tikzset{
 	 rect/.style={draw,align=center,rectangle, text width=8cm,rounded corners},	
 	 test/.style={draw, align=center,diamond, aspect=1.3, text width=8em}}
 \node [rect] (1) {Generate $\mathit{cTrace}$ for both $M_0$ and $M_1$};
 
  \node [rect,below=of 1] (12)     {k$\leftarrow$1};
 \node [rect,below=of 12] (2)     {\texttt{cbmc input.c -unwind k 
 --no-unwinding-assertions}};
 
 \node [test,below=of 2]  (t1)    {timeout?};
 
 \node [rect,below left=of t1]   (3)    {behaviors May Not be equivalent};
 
 \node [test,node distance=2cm and 1.5cm,below right=of t1]  (t2) 
 {\texttt{\_\_CPROVER\\\_assume} statement  SAT};
 
 \node [test,node distance=2cm and 4cm,below left=of t2]  (t3)    {User 
defined Assertion violated?};
 
 \node [rect,node distance=2cm and 3cm,below right=of t2]   (4)    {behaviors 
 May Not be equivalent};


\node [test,node distance=2cm and 4cm,below left=of t3]  (t21) {Verify 
 unwinding assertion};
  
 \node [rect,node distance=2cm and 5cm,below =of t21]   (6)  {Mark both path 
 as equivalent and proceed further};
 
 
 \node [test,node distance=2cm and 3cm,below right=of t3]  (t6)    {mismatch 
 in o/p values?};
 
 \node [rect,below left=of t6]  (r6)    {Report Not equivalent and provide CE 
 as a proof};
 
 \node [rect,node distance=2cm and 3cm,below right=of t6,] (5)     {Run two 
 programs over CE};
 
 \node [test,below=of 5]  (t4)    {mismatch in o/p values?};
 
 
 
 \node [rect,below right=of t4]   (7)  {Mark both path as equivalent and 
 proceed further};
 
 \node [rect,below left=of t4]  (8)    {Report Not equivalent and provide CE 
 as  a proof};
 
 
 \draw[->] (1) -- (12);
 \draw[->] (12) -- (2);
 \draw[->] (2) -- (t1);
 \draw[->]  (t1.180) -| node[pos=0.3,above]{Yes} node[pos=0.8,left]{\color{blue}{(Case~4)}} (3);
 \draw[->] (t1.0) -| node[above]{No} (t2.90);
 \draw[,->]  (t2.0) -| node[pos=0.3, above]{No}   node[pos=0.8,right]{\color{blue}{(Case~1)}} (4);
 \draw[->]  (t2.180) -| node[pos=0.3,above]{Yes} (t3);

 \draw[->]  (t21.270) -- node[pos=0.3,left]{Yes} node[pos=0.8,left]{\color{blue}{(Case~2)}} (6);
 \draw[->]  (t21.180) -- ($(t21)+(-6,0.0)$) |- 
 node[pos=0.3,above,rotate=90]{No, $k\leftarrow k+1$} (2);

 \draw[->]  (t3.0) -| node[pos=0.3,above,xshift=1.4cm]{Yes (Counter-example 
 (CE) exits)} node[pos=0.8,right]{\color{blue}{(Case~3)}}  
 (t6);
 
 \draw[->]  (t3.180) -| node[pos=0.3,above]{No} (t21);

 \draw[->]  (t6.180) -| node[pos=0.3,above]{Yes} node[pos=0.8,left]{\color{blue}{(Case~3.1)}} (r6);

 \draw[->]  (t6) -| node[pos=0.3,above]{No} (5);
 
 \draw[->] (5) -- (t4);
 \draw[->]  (t4.180) -| node[pos=0.3,above]{Yes} node[pos=0.8,left]{\color{blue}{(Case~3.3)}} (8);
 \draw[->]  (t4.0) -| node[pos=0.3,above]{No} node[pos=0.8,right]{\color{blue}{(Case~3.2)}} (7);
 \end{tikzpicture}
 }
\caption{Control flow graph of counter-example generation using CBMC and its utilization in a PBEC framework.}
\label{Fig:CFG}
\end{figure*}

To obtain the counter-example, i.e., assigning 
suitable value to the inputs, we rely on CBMC \cite{Clarke04CBMC} . 
Specifically, for a given upper bound, CBMC verifies the specified  assertions. If any violation of an
assertion is detected, a counter-example is generated. Let us consider the 
\textit{cTraces} as shown in Fig.~\subref*{SubFig:CM0} and Fig.~\subref*{SubFig:CM1}. The input to the CBMC in C 
for this case is shown in Fig~\ref{Fig:CBMCInput}. 

The variables appearing in the \textit{cTrace} of $M_0$
(Fig.~\subref*{SubFig:CM0}) are suffixed with $\mathtt{\_s}$,whereas 
the variables appearing in the \textit{cTrace} of $M_1$ (Fig.~\subref*{SubFig:CM1}) are
suffixed with $\mathtt{\_t}$. Since program equivalence entails identical output(s)
generated by the two programs when fed with the same input(s), the input variable
$n$ is not suffixed with either $\mathtt{\_s}$ or $\mathtt{\_t}$. Lines~3 and 4
declare the variables appearing in the \textit{cTrace} of $M_0$ and the
\textit{cTrace} of $M_1$, respectively, along with their data type which is
integer for all the variables. The lines 8--16 and 18--26 capture the data
transformations and the conditions of execution of the paths appearing in the
\textit{cTrace} of the $M_0$ and $M_1$, respectively. We use
\texttt{\_\_CPROVER\_assume} statements to allow only those computation
that satisfy a given condition. For example CBMC first picks the value for $n$
non-deterministically from the domain of integers. The statement
\texttt{\_\_CPROVER\_assume($n\geq 0$)} at line~5 further restricts the range
of $n$ for all program computations to be greater than or equal to 0. Note that
if there is no computation satisfying the condition, say $P$, mentioned in
\texttt{\_\_CPROVER\_assume} statement, then all the assertions hold
vacuously. We check this by adding \texttt{assert($!P$)} statement after each
\texttt{\_\_CPROVER\_assume} statement so that if one of the
\texttt{assert($!P$)} statement is true then we declare that all the possible
computations represented by \textit{cTrace} are false computations
\cite{Chouksey17} i.e., they never execute. Finally, we check the equivalence
of the live variables  ($x\_s,y\_s,x\_t,y\_t$) and output variables ($\mathit{out\_s}$,
$\mathit{out\_t}$) using  the \texttt{assert} statements (lines
\texttt{27}--\texttt{29}).

CBMC is able to automatically determine an upper bound on the number of loop iterations
in many cases. It  may fail if the number of loop iterations is highly data-dependent.
Therefore, to verify the assertions with CBMC we use the following command:
\texttt{cbmc fileName.c -unwind k --no-unwinding-assertions} where 
\textit{fileName.c} is the name of the target program, $k$ is the bound on the 
number of iterations of the loop in the program called as Unwinding Loop 
Bound (ULB) and \texttt{--no-unwinding-assertions} disables the unwinding assertion 
check and changes the unwinding assertion to an unwinding assumption. We use the 
option \texttt{--no-unwinding-assertions} so that a counter-example 
might be found within the small state space generated with the small ULB. If 
the target program contains a loop then CBMC unwinds the loop $k$ times 
and check the properties. Note that if there are multiple loops in the program, 
the bound $k$ applies to all loops. A violation of the property is reported if 
it is found within $k$ ULB and CBMC will give a counter-example. Otherwise, we 
iteratively run CBMC with increasing ULBs for the loops until an 
assertion violation is found or a given time limit is reached. 
\begin{comment}
There are two 
possible outcomes by CBMC for a given ULB.
\begin{itemize}
\item \textit{Counter-example:} The CBMC reports a counter-example, i.e., it 
has 
found input data within a given ULB for which the user-defined assertion is not 
valid.
\item \textit{No counter-example:} There are two possibilities when CBMC does not 
report a counter-example for a given ULB.
\begin{itemize}
\item CBMC hits the time limit and no counter-example has been found. In 
this case it is possible that the user-defined assertions are indeed valid  
or the counterexamples are too complex to be found within a given time limit.
\item CBMC does not hit the time limit and does not report a counter-example. 
In this case, if unwinding assertion fails (i.e., the loop can execute more then 
ULB times) then we execute CBMC with increasing number of ULBs. Otherwise, CBMC 
proved assertions to be valid, it indeed was.
\end{itemize}
\end{itemize}

Note that if no counterexample is found with depth $k$ then unwinding assertion 
is verified 
along with the user defined assertion.
if unwinding assertion fails then a violation of the property could still occur 
in further iterations of the loop. Hence  Otherwise CBMC proved a property to 
be valid, it indeed 
was.

we used 1000 second timeout for CBMC.
For our experiment we used CBMC (built from revision 4503,
used with Z3 as the decision procedure)

No counterexample The tool has hit the time limit and no counterexample has 
been found. Thus, either the solution is correct or the counterexamples are too 
complex to be found within the available time.
\end{comment}

\section{Incorporation of Results in Equivalence Checking Framework\label{Sec:incorpoResult}}
The PBECs are sound but not complete. Therefore, all the PBECs 
report that the behaviors ``May Not be equivalent'' once
they fail to prove the equivalence of source and transformed behaviors. Using
the output of CBMC, we can actually make the PBEC more
powerful. In some scenarios, the PBEC can report that the
behaviors are ``Not equivalent'' (instead of ``May Not") along with a
counter-example. Also, in some scenarios, the non-equivalence result reported by
the PBEC can be proved to be a \textit{false negative} and equivalence
checking will proceed further. In the following, we discuss how we can
incorporate the CMBC result to improve the equivalence checking framework.
 \begin{itemize}
  \item {\textbf{Case~1:} \it One of the conditions mentioned in 
  \texttt{\_\_CPROVER\_assume} statement is not 
  satisfiable}: In this case, we report to PBEC that all
  the possible computations represented by \textit{cTrace} are false 
  computations.  Consequently, we need to proceed further in the 
  equivalence checking process.
 \item {\textbf{Case~2:} \it The unwinding assertions are valid and CBMC 
 does not find any counter-example}: This means the values of all the live 
 variables and output variables are the same for both \textit{cTraces}. So the non-equivalence 
 reported by the PBEC may be a false negative. In this case, we need to proceed 
 further in equivalence checking by declaring the corresponding path pair 
 ($\alpha,\beta$) as an equivalent path. This actually helps the PBEC to avoid 
 false negative results during the course of equivalence checking.
 \item {\textbf{Case~3:} \it CBMC reports counter-example for some variables:} 
 This means the data transformation of some variables is not equivalent 
 in the \textit{cTraces}. 
 
 {\textbf{Case~3.1:} \it A mismatch is found for an output variable}:   
 This is surely a non-equivalence case. So the equivalence checker correctly 
 found the non-equivalence of the behaviors. In this case, the PBEC 
 reports that the behaviors are ``Not equivalent'' along with the 
 counter-example. 

 If a mismatch is found only for live variables (which are 
 not output variables), then we cannot conclude definitely that the final 
 outputs of both the behaviors will not be the same. There may be some other 
 operations in the subsequent execution of the FSMDs which will make the 
 behaviors equivalent. Therefore, we need to execute the two programs with the 
 counter-example produced by CBMC and check if the outputs of the two 
 programs are the same or not. 
 
{\textbf{Case~3.2:} \it The outputs are the same}: This is not a 
 non-equivalent case. Consequently, we need to proceed further in the 
 equivalence checking process. 

{\textbf{Case~3.3:} \it The outputs of the two 
 programs are not the same}: This is surely a non-equivalence scenario; 
 in this case, the equivalence checker will report the behaviors are ``Not 
 equivalent'' along with the counter-example.
 \item {\textbf{Case~4:} \it CBMC hits the time limit:} In this case, CBMC  
 has failed to generate a counter-example because of time out. 
 So no counter-example will be provided to the user. The PBEC reports the behaviors ``May Not be equivalent''. 
 \end{itemize}

\section{Overall equivalence checking framework\label{Sec:EVP+CE}}
\begin{algorithm}[!t]
%\small
\scriptsize
 \SetArgSty{textup} % to avoid italic font 
 %\SetKwInOut{Input}{Input}
 %\SetKwInOut{Output}{Output} 
% \Input{Two FSMD $M_0$ and $M_1$, a path $\alpha$ of a path cover of $M_0$, a 
% path $\beta$ of a path cover of $M_1$, $\mathtt{EQ\_LIST}$ contains equivalent 
% path pairs, $\mathtt{C\_LIST}$ contains candidate for conditionally equivalent 
% path pairs.}
% \Output{$\bar{v}=\langle v_1,v_2,\dots,v_n\rangle$ input variable list such 
% that $v_i$ represents the value of the input variable $v_i$. A 
%  boolean variable $\mathit{Equiv}$ which is $\mathtt{True}$ if the 
 % $\alpha\simeq\beta$ otherwise $\mathtt{False}$. A boolean variable 
 % $\mathit{flaseComp}$ which is $\mathtt{True}$ if all computations 
 % represented by \textit{cTrace} are false computation otherwise 
 % $\mathtt{False}$.}
 DFS from the start state of $\alpha$ in $\mathtt{C\_LIST}$ to obtain the 
 sequence  $\langle p_{0j},~p_{0j+1},~...,~p_{0k},~\alpha\rangle$.\label{Line:ctraceM0}\\
 DFS from the start state of $p_{0j}$ in $\mathtt{EQ\_LIST}$ to obtain the sequence 
 $\langle p_{00},~p_{01},~...,~p_{0i}\rangle$.\label{Line:ctraceM1}\\
 Encode the $\mathit{cTrace}=\langle 
 p_{00},p_{01},\dots,p_{0i},p_{0j},p_{0j+1},\dots,p_{0k},\alpha\rangle$ 
 and its corresponding \textit{cTrace} in $M_1$ as C, say ``input.c".\label{Line:encode}\\
 Initialize the unwinding loop bound (ULB) $k$ to 1.\\
 Use \texttt{cbmc input.c -unwind k --no-unwinding-assertions} command to 
 invoke  CBMC.\\
\uIf{The condition mentioned in  \texttt{\_\_CPROVER\_assume} is not satisfiable\label{Line:caseStart}}
  {{\KwRet{$\langle\text{NULL},\mathtt{False},\mathtt{True}\rangle$}\tcc*{Case~1}}}
\uElseIf{All the unwinding assertions along with the user defined assertions are 
valid}
 {{\KwRet{$\langle\text{NULL},\mathtt{True},\mathtt{False}\rangle$}\tcc*{Case~2}}}
 \uElseIf{ CBMC produces a counter-example for the assertion belongs to an output 
 variable}
  {{\KwRet{$\langle \bar{v},\mathtt{False},\mathtt{False}\rangle$}\tcc*{Case~3.1}}}
 \uElseIf{CBMC produces a counter-example for the assertion belongs to live 
  variable}
  {Execute both $M_0$ and $M_1$ with the values obtained from CBMC as inputs.\\
     \uIf{outputs are the same}
 {{\KwRet{$\langle \text{NULL},\mathtt{False},\mathtt{False}\rangle$}}\tcc*{Case~3.2}}
     \Else
  {{\KwRet{$\langle \bar{v},\mathtt{False},\mathtt{False}\rangle$}}\tcc*{Case~3.3}}
    }  
 \uElseIf{CBMC hits the time limit\label{Line:caseEnd}}
  {{\KwRet{$\langle \text{NULL},\mathtt{False},\mathtt{False}\rangle$}}\tcc*{Case~4}}
 \Else
  { Increase ULB by one (i.e., k=k+1) and go to step~5}
 %\ENDIF
 \caption{$\mathtt{counterExmapleGenerator}$($M_0$, $M_1$, $\alpha$,  $\beta$, 
 $\mathtt{EQ\_LIST, C\_LIST}$)}
 \label{Algo:ctrace}
 \end{algorithm}

 
 %---------------The VP Algo----------------------------
 	\begin{algorithm}[!t]
  \scriptsize
 		\SetArgSty{textup} % to avoid italic font 
 		\ForEach{path $\alpha:(q_{0i}\Rightarrow q_{0m})$ in $P_0$} 
 		{	\uIf
 			{
 				path $\beta:(q_{1j}\Rightarrow q_{1n})$ can be found in
 				$P_1$ such that $\alpha\simeq\beta$
 			} 
 			{
 				$W_{csp}=W_{csp}\cup \{(q_{0m},q_{1n})\}$;\\ 
 				Insert $(\alpha,\beta)$ in EQ\_LIST.
 			}
 			\uElseIf
 			{
 				path $\beta:(q_{1j}\Rightarrow q_{1n})$ can be found in $P_1$
 				such that $\alpha\simeq_c\beta$
 			} 
 			{
 				\uIf
 				{
 					$q_{0m}$ or $q_{1n}$ is reset state
 				} 
 				{
 					\KwRet{\textit{failure}}; 
 				}			
 				\Else 
 				{
 					Insert $(\alpha,\beta)$ in C\_LIST.\\
 					$\mathtt{correspondenceChecker}(M_0,M_1,q_{0m},q_{1n},\newline P_0,P_1, 
 					W_{csp})$\;
 						
 				}	
 			} 
 			\Else 
 			{ 
 				$\langle\bar{v},Equiv,falseComp\rangle$$\leftarrow$ 
 				$\mathtt{counterExmapleGenerator}$($M_0$, $M_1$, $\alpha$,  $\beta$, 
 			 $\mathtt{EQ\_LIST, C\_LIST}$)\label{Line:counter}\;
 			 \uIf{$\mathit{falseComp}==\mathtt{True}$}
 			 {Proceed Further  \tcc{Case~1}}
 			 \uElseIf{$\bar{v}\neq\mathtt{NULL}$}
 			 {	\KwRet{\textit{Not equivalent}}; \tcc{Case~3.1}}
			\uElseIf{$\bar{v}==\mathtt{NULL}$ and $\mathit{Equiv}==\mathtt{True}$}
 				{Proceed Further \tcc{Case~2}} 
 				\label{step:fail}
  	\uElseIf{$\bar{v}==\mathtt{NULL}$ and $\mathit{Equiv}==\mathtt{False}$}
 				{Proceed Further \tcc{Case~3.2}} 
 				\label{step:fail}
 			\Else	
 			 {\KwRet{\textit{May Not be Equivalent}}; \tcc{Case~4}}
 			}
 					
 		} 
  $\mathtt{EQ\_LIST}=\mathtt{EQ\_LIST}\cup\{\text{Last member of }\mathtt{C\_LIST\}}$\\
  $\mathtt{C\_LIST}=\mathtt{C\_LIST}\setminus\{\text{Last member of }\mathtt{C\_LIST}\}$\\ 
 		\KwRet{\textit{success}}; 
 	 \caption{$\mathtt{correspondenceChecker}(M_{0},M_{1},q_{0i},q_{1j},\newline,
 	 P_0,P_1,W_{csp})$}
 		\label{Algo:correspondenceChecker} 
 	\end{algorithm}
 
The abstract version of our counter-example generation  represented by the
function $\mathtt{counterExmapleGenerator}$ is presented in Algorithm~\ref{Algo:ctrace}.  The control flow of
Algorithm~\ref{Algo:ctrace} is given in Fig.~\ref{Fig:CFG}.
The function $\mathtt{counterExmapleGenerator}$ takes as input two FSMDs $M_0$ and $M_1$, a
path $\alpha$ from the path cover of $M_0$, a  path $\beta$ from the path cover of
$M_1$, $\mathtt{EQ\_LIST}$ contains equivalent path pairs and
$\mathtt{C\_LIST}$ contains candidates for conditionally equivalent path pairs.
The function $\mathtt{counterExmapleGenerator}$ returns
$\langle\bar{v},\mathit{Equiv},\mathit{falseComp}\rangle$, where $\bar{v}=\langle
v_1,v_2,\dots,v_n\rangle$ is the input variable list such  that $v_i$ represents the
value of the input variable $v_i$, $\mathit{Equiv}$ is
$\mathtt{True}$ if $\alpha\simeq\beta$ and $\mathtt{False}$ otherwise and 
$\mathit{falseComp}$ is $\mathtt{True}$ if all the
computations represented by \textit{cTrace} are false computations and 
$\mathtt{False}$ otherwise. In lines~\ref{Line:ctraceM0}--\ref{Line:ctraceM1} of
Algorithm~\ref{Algo:ctrace}, a \textit{cTrace} is constructed from the
$\mathtt{EQ\_LIST}$ and $\mathtt{C\_LIST}$ as discussed in Sec.~\ref{Sec:CTrace}. The
\textit{cTrace} is encoded as input to CBMC at line~\ref{Line:encode}. The
output generated by CBMC may result in various scenarios as discussed in
Sec.~\ref{Sec:incorpoResult}.  Lines~\ref{Line:caseStart}--\ref{Line:caseEnd}
of Algorithm~\ref{Algo:ctrace} handle  these cases.

The enhanced version of $\mathtt{correspondenceChecker}$ function of the EVP method \cite{Chouksey18} after incorporating
 the result of the function $\mathtt{counterExmapleGenerator}$ is presented in 
Algorithm~\ref{Algo:correspondenceChecker}. In case of failure,
Algorithm~\ref{Algo:correspondenceChecker} invokes the function
$\mathtt{counterExmapleGenerator}$ (Algorithm~\ref{Algo:ctrace}) at
line~\ref{Line:counter}. It may be noted that the EVP method reports failure under this scenario.
  If $\mathtt{counterExmapleGenerator}$
returns a counter-example (i.e., $\bar{v}\neq \mathtt{NULL}$) then the function
$\mathtt{correspondenceChecker}$ returns ``Not equivalent" i.e., the two FSMDs are
not equivalent (line~17). If CBMC hits the time limit then  we
cannot decide whether $M_0$ is equivalent to  $M_1$. Hence the function
$\mathtt{correspondenceChecker}$ returns ``May Not be Equivalent" (line~23). If
CBMC reports that all the possible computations represented by \textit{cTrace}
are false computations (i.e., the variable $\mathit{falseComp}$ is
$\mathtt{True}$) then the function $\mathtt{correspondenceChecker}$ needs to be
modified to handle this scenario (line~15). 
If CBMC finds the mismatch in the values of a live variable but outputs of the
two programs are the same then we do not report the counter-example (line~21). To handle this 
case also $\mathtt{correspondenceChecker}$ needs to be modified. 
If CBMC declares that the path pair
($\alpha,\beta$) are equivalent (i.e., the variable $\mathit{Equiv}$ is
$\mathtt{True}$) then it is a false negative result of the
$\mathtt{correspondenceChecker}$ function (line~19). The
$\mathtt{correspondenceChecker}$ function must take some decision to avoid the
false negative case in the future.

\section{Experimental Result\label{Sec:Exp}}
\begin{frame}[t]{Experimental Results}
\begin{table}[!t]
\setlength\tabcolsep{5pt} % default value: 6pt
\renewcommand{\arraystretch}{1.2}
  \centering
  \begin{threeparttable}
  \begin{tabular}{|l|c|c|c|c|c|c|c|c|}
    \hline
    \multirow{2}{*}{Benchmarks}&
    \multirow{2}{*}{\#Path}&
    \multicolumn{2}{c|}{\#State} &
    \multicolumn{2}{c|}{Decision} &
    \multicolumn{2}{c|}{Time (ms)} &
    \multirow{2}{*}{Lines}\\\cline{3-8}
    &&$M_0$&$M_1$&EVP&Our&EVP&Our&
      \only<1,3-4>{\\\hline}
      \only<2>{\\\hline\rowcolor{blue!20}}DIFFEQ&3&15&9&E&E&25&25&-% 
      \only<1,3-4>{\\}
      \only<2>{\\\rowcolor{blue!20}}LRU&39&33&32&E&E&1038&1038&-%
      \only<1-2,4>{\\}
      \only<3>{\\\rowcolor{blue!20}}DCT&1&8&16&MNE&NE&85&766&185%output change
      \only<1-2,4>{\\}
      \only<3>{\\\rowcolor{blue!20}}PERFECT&7&6&4&MNE&NE&56&227&74%live Variable change
      \only<1-2,4>{\\}
      \only<3>{\\\rowcolor{blue!20}}MODN&9&8&9&MNE&NE&66&890&137%live Variable Change
      \only<1-2,4>{\\}
      \only<3>{\\\rowcolor{blue!20}}GCD&11&8&4&MNE&NE&31&100&97%
      \only<1-3>{\\}
      \only<4>{\\\rowcolor{blue!20}}Test Case&6&5&5&MNE&MNE&20&26&32%
   \\\hline
  \end{tabular}
    \begin{tablenotes}
  \centering
    \tiny
    \item E -- Equivalent, MNE -- May Not be Equivalent, NE -- Not Equivalent
    \end{tablenotes}
  \end{threeparttable}
  \end{table}%
  \begin{overlayarea}{\textwidth}{3cm} 
   \begin{itemize}
   \only<1>{\item Implemented CEG on top of the EVP.}
   \only<2>{\item No side effect on the existing method.}
   \only<3>{\item The EVP takes strong decisions about the non-equivalence of behaviors.}
   \only<4>{\item Finds a scenario where the EVP gives false negative result.}  
   \end{itemize}
\end{overlayarea}
\end{frame}

\section{Related Work\label{Sec:Rework}}
\input{texFiles/relatedwork.tex}
\section{Conclusion\label{Sec:conclusion}}
\begin{frame}{Conclusion \& Future Works}
\begin{itemize}
\item Proposed a CEG mechanism for the PBEC.
\item PBEC is further strengthened with the CEG mechanism.
\item For some scenarios PBEC reports not equivalent and provide CE as a proof.
\item Identified a false negative result of the EVP method.
\item Similar CEG mechanism can also be developed for other reported equivalence checking methods as well.
\item Enhance the EVP to handle false negative cases.
\end{itemize}
\end{frame}

\section*{Acknowledgment}
This work is
partially supported by DST, Govt.\ of India (Project code: ECR/2017/000492).
The authors would like to acknowledge Dr.\ Kunal Banerjee, currently 
at Intel Labs, Bangalore, India, for his useful insights during the formulation of this work. 
\bibliographystyle{IEEEtran}
\bibliography{IEEEabrv,allShortReference} 
\end{document}

